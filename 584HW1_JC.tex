\documentclass[11pt,letterpaper]{article}
\usepackage[top=.5in,textheight=9in]{geometry}
\usepackage{amsmath, amsthm, amssymb}
\usepackage{enumerate}
\usepackage{xfrac}


% Everything after a % sign is commented out.
% This is sometimes useful to write notes to yourself
% or to add spacing in the tex file so that it is easier
% to read.


% Some useful `macros'
% % % Feel free to define your own!
\newcommand{\C}{\mathbb{C}}
\newcommand{\N}{\mathbb{N}}
\newcommand{\R}{\mathbb{R}}
\newcommand{\Z}{\mathbb{Z}}

\newcommand{\cS}{\mathcal S}


% Here is a pretty way to write down the problem
\newtheorem{innerprob}{Problem}
\newenvironment{prob}[1]
  {\renewcommand\theinnerprob{#1}\innerprob}
  {\endinnerprob}
% Here is a pretty way to wrote down the solution
\newenvironment{solution}
  {\renewcommand\qedsymbol{}\begin{proof}[Solution]}
  {\end{proof}\bigskip}

\setlength\parindent{0cm}
\setlength\parskip{5pt plus 1pt minus 1pt}


\title{Assignment \#1\\Math 584A}
\author{
	John Cohen
	}
\date{Due September 10th at 1 am (via Gradescope)}









\begin{document}

\maketitle

For uploading to Gradescope, it will be easiest to put each solution on a different page.  The code for this is commented out in the tex file.

% Don't write anything between \begin{document}
% and \maketitle or it will show up before your name
% and the rest of the title stuff.



%State the problem
%\begin{prob}{PROBLEM #}
%  WRITE PROBLEM
%\end{prob}

% Note prob is custom-made above will



\begin{prob}{1}  % `prob' starts the (custom made, above) problem
			%  and the 
Fix any two points $x,y\in \R^n$.  Show that
\[
	\lim_{p\to\infty} d_p(x,y) = d_\infty(x,y).
\]
\end{prob}
%Uncomment the lines below to solve the problem
\begin{solution}
	Let $i\leq n$ be defined such that $|x_i-y_i| \geq |x_m-y_m|$ for all $m\leq n$. By definition, $|x_i-y_i| = d_\infty(x,y)$. Notice that $$|x_i-y_i| \leq |x_1-y_1| + \dotsc + |x_i-y_i|+ \dotsc + |x_n-y_n|\leq n|x_i-y_i|.$$ Thus, $$|x_i-y_i|^p \leq |x_1-y_1|^p + \dotsc + |x_i-y_i|^p+ \dotsc + |x_n-y_n|^p\leq n|x_i-y_i|^p.$$ By taking each expression in the inequality to the $1/p$ power, we get $$(|x_i-y_i|^p)^{\frac{1}{p}} \leq (|x_1-y_1|^p + \dotsc + |x_i-y_i|^p+ \dotsc +|x_n-y_n|^p)^{\frac{1}{p}} \leq (n|x_i-y_i|^p)^{\frac{1}{p}},$$ and
	\begin{equation}\label{e1}
		|x_i-y_i| \leq d_p(x,y) \leq n^\frac{1}{p}|x_i-y_i|.
	\end{equation}  
	Taking the limit as p approaches infinity gives $$\lim_{p\to\infty}|x_i-y_i| \leq \lim_{p\to\infty}d_p(x,y) \leq \lim_{p\to\infty}n^\frac{1}{p}|x_i-y_i|.$$ Thus, $$d_\infty(x,y) \leq \lim_{p\to\infty}d_p(x,y) \leq d_\infty(x,y).$$ Therefore, by the trichotomy,
	\[
	\lim_{p\to\infty} d_p(x,y) = d_\infty(x,y).
	\]
\end{solution}
\newpage






\begin{prob}{2}  % `prob' starts the (custom made, above) problem
			%  and the 
Fix $p_1,p_2 \in [1,\infty]$ such that $p_1 < p_2$ (we allow the case $p_2 = \infty$).  Find constants $\underline C, \overline C >0$ such that
\[
	\underline C d_{p_1}(\bar x, \bar y)
		\leq d_{p_2}(\bar x, \bar y)
		\leq \overline C d_{p_1}(\bar x, \bar y),
\]
for all $x,y \in \R^N$.  Hint: you may find it helpful to do the case $p_1 = 1$ and $p_2=\infty$ first.
\end{prob}
%Uncomment the lines below to solve the problem
\begin{solution}
	The corresponding $d_{p_1}(\bar x, \bar y)$ for $p_1$. Let $d_\infty(\bar x, \bar y) = |x_i-y_i|$ for some $i\leq N$. By \eqref{e1}, we see that $$n^{-\frac{1}{p_1}}d_{p_1}(x,y) \leq |x_i-y_i|.$$ By setting $\underline C = n^{-\frac{1}{p_1}}$, we get $\underline C d_{p_1}(\bar x,\bar y) \leq d_\infty(\bar x, \bar y)$. Since the choice of $p_1\geq 1$ was arbitrary, this is valid for any $d_p(\bar x, \bar y)$.
	
	For the same $d_\infty(\bar x, \bar y)$, factoring out $|x_i-y_i|^{p_1}$ from the expression for $d_{p_1}(\bar x, \bar y)$ gives 
	\[\begin{split}
	d_{p_1}(\bar x, \bar y) &= \left [ |x_i-y_i|^{p_1} \left ( \frac{|x_1-y_1|^{p_1}}{|x_i-y_i|^{p_1}} + \dotsc + \frac{|x_i-y_i|^{p_1}}{|x_i-y_i|^{p_1}} + \dotsc \frac{|x_N-y_N|^{p_1}}{|x_i-y_i|^{p_1}} \right ) \right ] ^{1/{p_1}} \\
	&=|x_i-y_i| \left [ \left ( \frac{|x_1-y_1|}{|x_i-y_i|} \right )^{p_1} + \dotsc + 1 + \dotsc \left ( \frac{|x_N-y_N|}{|x_i-y_i|} \right )^{p_1} \right ] ^{1/{p_1}}.
	\end{split}
	\]
	Let $a_m = \frac{|x_m-y_m|}{|x_i-y_i|}$ for each $m\leq N$. Since $|x_i-y_i| \geq |x_m-y_m|$, $a_m<1$ for all $m \leq N$. Thus, 
	\begin{equation}\label{e2}
		d_{p_1}(\bar x, \bar y) = |x_i-y_i| (a_1^{p_1} + \dotsc + 1 + \dotsc + a_N^{p_1})^{1/p_1}.
	\end{equation}
	Since we set $d_\infty(\bar x, \bar y) = |x_i-y_i|$, $$d_{p_1}(\bar x, \bar y) = d_\infty(\bar x, \bar y) (a_1^{p_1} + \dotsc + 1 + \dotsc + a_N^{p_1})^{1/p_1}.$$ Because $a_1^{p_1} + \dotsc + 1 + \dotsc + a_N^{p_1} > 1$, $(a_1^{p_1} + \dotsc + 1 + \dotsc + a_N^{p_1})^{1/p_1} > 1$. Therefore, $$d_\infty(\bar x, \bar y)\leq d_{p_1}(\bar x, \bar y).$$ Thus, 
	\begin{equation}\label{e3} 
		\underline C d_{p_1}(\bar x, \bar y) \leq d_\infty(\bar x, \bar y)\leq \overline C d_{p_1}(\bar x, \bar y) 
	\end{equation}	
	where $\underline C = n^{-\frac{1}{p_1}}$ and $\overline C = 1$.
	
	Consider the $p_2$ metric on the same vectors $\bar x$ and $\bar y$. Equation \eqref{e2} for $p_2$ gives $$d_{p_2}(\bar x, \bar y) = |x_i-y_i| (a_1^{p_2} + \dotsc + 1 + \dotsc + a_N^{p_2})^{1/p_2}.$$ Since $p_2 > p_1$ and $a_m < 1$, $a_m^{p_2}\leq a_m^{p_1}$ and $$(a_1^{p_2} + \dotsc + 1 + \dotsc + a_N^{p_2})^{1/p_2} \leq a_1^{p_1} + \dotsc + 1 + \dotsc + a_N^{p_1})^{1/p_1}.$$ Therefore, $$d_{p_2}(\bar x, \bar y) \leq d_{p_1}(\bar x, \bar y).$$ Combining this with \eqref{e3} gives $$n^{-\frac{1}{p_1}}d_{p_1}(\bar x, \bar y) \leq d_\infty(\bar x, \bar y) \leq d_{p_2}(\bar x, \bar y) \leq d_{p_1}(\bar x, \bar y).$$ Thus, $$\underline C d_{p_1}(\bar x, \bar y) \leq d_{p_2}(\bar x, \bar y)
	\leq \overline C d_{p_1}(\bar x, \bar y)$$ where $\underline C = n^{-\frac{1}{p_1}}$ and $\overline C = 1$.
	
\end{solution}
\newpage






\begin{prob}{3}  % `prob' starts the (custom made, above) problem
			%  and the 
Consider the metric space $(X, d_{\rm disc})$, where $X$ is a nonempty set.  Suppose that $(x_n)_n$ is a sequence in $X$.  Show that
\[
	\lim_{n\to\infty} x_n = x
\]
if and only if there exists $N$ such that $x_n = x$ for all $n \geq N$.
\end{prob}
%Uncomment the lines below to solve the problem
\begin{solution}
	Fix any $\epsilon$ such that $0 < \epsilon < 1$. Since $\lim_{n\to\infty} x_n = x$, there must exist $N \in \N $ such that $d_{disc}(x_n,x)< \epsilon$ whenever $n\geq N$. Since we are in the discrete metric, $d_{disc}(x_n,x)< 1$ only when $x_n = x$. Therefore, $x_n = x$ for all $n\geq N$.
	
	Assume there exists some $N\in \N$ such that for all $n\geq N$, $x_n = n$. Thus, for all $n\geq N$, $d_{disc}(x_, x)  = 0$. Therefore, $d_{disc}(x_, x) < \epsilon$ for any $\epsilon > 0$. Thus, $\lim_{n\to\infty} x_n = x$.
\end{solution}
\newpage



\begin{prob}{4}  % `prob' starts the (custom made, above) problem
			%  and the 
Prove Young's inequality: for any $p,q >1$ such that
\[
	1
		= \frac{1}{p} + \frac{1}{q}
\]
for any $x,y\in \R$,
\[
	xy \leq \frac{|x|^p}{p} + \frac{|y|^q}{q}.
\]
Hint: for any fixed $y \geq 0$, consider the function $f: [0,\infty) \to \R$, defined by
\[
	f(x) = \frac{x^p}{p} + \frac{y^q}{q} - xy,
\]
and show that the minimum of $f$ is zero.
\end{prob}
%Uncomment the lines below to solve the problem
\begin{solution}
	Consider first the function $f:[0,\infty)\to \R$ defined by
	\[
	f(x) = \frac{x^p}{p} + \frac{y^q}{q} - xy.
	\]
	For a fixed $y\geq 0$, we will show that the minimum of this function is zero, and thus  
	\[
	\frac{x^p}{p} + \frac{y^q}{q} > xy.
	\]
	Checking first the bounds at $x=0$ and the limit as $x\to\infty$ gives $$f(0) = \frac{y^q}{q}\quad \text{ and }\quad \lim_{x\to\infty} f(x) = \infty.$$ Since $y\geq 0$, $f(0)\geq 0$.
	
	Taking the derivative and second derivative of $f(x)$ with respect to $x$ gives $$f^\prime(x)= x^{p-1}-y \text { and } f^{\prime \prime} = (p-1)x^{p-2}.$$ Solving for $f^\prime(x) = 0$ gives a critical point at $x = y^{\frac{q}{p}}$. Since $p>1$ and $x\geq 0$, $f^{\prime \prime} \geq 0$ for all $x\in [0,\infty)$. This critical point is therefore, a local minimum. Evaluating $f$ at this value of $x$ gives
	\[\begin{split}
		f(x = y^{\frac{q}{p}}) &= \frac{(y^{\frac{q}{p}})^p}{p} + \frac{y}{q} -y^{\frac{q}{p}}y\\
		&= \frac{y^q}{p}+\frac{y^q}{q}-y^{\frac{q+p}{p}}\\
		&= y^q \left ( \frac{1}{p} + \frac{1}{q} \right ) - y^{\frac{q+p}{p}}\\
		&=y^q - y^{\frac{q+p}{p}}
	\end{split}	
	\]
	To continue, we need a rearrangement of Young's Inequality.
	\[\begin{split}
		1 &= \frac{1}{p} + \frac{1}{q}\\
		&= \frac{q}{pq} + \frac{p}{pq}\\
		&= \frac{p+q}{pq}
	\end{split}
	\]
	Thus,
	\[\begin{split}
		f(x = y^{\frac{q}{p}}) &= y^q - y^{q(\frac{q+p}{pq})}\\
		&= y^q - y^{q(1)}\\
		& = 0.
	\end{split}	
	\]
	Therefore, zero is the absolute minimum of this function. Since an identical argument can be made for $y$, $f(x)\geq 0$ for all $x,y\geq 0$. In this case, $x = |x|$ and $y = |y|$, and $$|x||y| \leq \frac{|x|^p}{p} + \frac{|y|^q}{q}.$$ Finally, by Cauchy-Schwarz inequality, $$xy \leq \frac{|x|^p}{p} + \frac{|y|^q}{q}.$$ for any $x,y \in \R$.
	
\end{solution}
\newpage



\begin{prob}{5}  % `prob' starts the (custom made, above) problem
			%  and the 
Fix $N\geq 1$ and $p>1$.  Let $q$ be as in problem 4. Show the following:
\begin{enumerate}[(i)]

	\item Show that, for any $\bar x, \bar y \in \R^N$, we have
		\[
			x\cdot y \leq \frac{|x|_p^p}{p} + \frac{|y|_q^q}{q}.
		\]
		Recall that
		\[
			|x|_p = \left(|x_1|^p + \cdots + |x_N|^p\right)^{\sfrac1p},
		\]
		and similarly for $|\cdot|_q$.  
	
	\item Show that $d_p$ is a metric on $\R^N$.  You may find it helpful to write, for any $i=1,\dots, N$,
		\[
			|x_i - y_i|^p = (x_i - y_i) z_i
		\]
		for a well-chosen $z_i$.
	
\end{enumerate}
\end{prob}
%Uncomment the lines below to solve the problem
\begin{solution}[Part (i)]
	 The dot product of the vectors $\bar x$ and $\bar y$ is $$\bar x \cdot \bar y = x_1y_1 + \dotsc + x_Ny_N.$$ Since $p$ and $q$ satisfy Young's Inequality, $$x_ny_n \leq \frac{|x_n|^p}{p} + \frac{|y_n|^q}{q}.$$ for $n\leq N$. Thus, $$\bar x \cdot \bar y \leq \left ( \frac{|x_1|^p}{p} + \frac{|y_1|^q}{q} \right ) + \dotsc + \left( \frac{|x_N|^p}{p} + \frac{|y_N|^q}{q}\right ).$$ Combining the terms containing $x$ and the terms containing $y$ gives 
	 \[\begin{split}
	 \bar x \cdot \bar y &\leq \left ( \frac{|x_1|^p+ \dotsc + |x_N|^p}{p} \right ) + \left( \frac{|y_1|^q+\dotsc+|y_N|^q}{q}\right )\\
	 &\leq \frac{|\bar x|_p^p}{p} + \frac{|\bar y|_q^q}{q}.
	 \end{split}\]
	 
\end{solution}

\begin{solution}[Part (ii)]
	To show that $d_p$ is a metric on $\R ^N$, recall the definition of $d_p$ for any $p>1$: $$d_p(\bar x, \bar y) = \left(|x_1-y_1|^p + \cdots + |x_N-y_n|^p\right)^{\sfrac1p}.$$ Fix any $n\leq N$. Since, $|x_n-y_n| = 0$ only when $x_n = y_n$, $|x_n-y_n|^p = 0$ only when $x_n = y_n$. Since the choice of $n$ was arbitrary $|x_1-y_1|^p + \cdots + |x_N-y_n|^p = 0$ only if $x_n = y_n$ for all $n\leq N$. The same can also be said about that sum raised to the $\sfrac{1}{p}$ power, thus satisfying positive definiteness.
	
	Consider $$d_p(\bar y, \bar x) = \left(|y_1-x_1|^p + \cdots + |y_N-x_n|^p\right)^{\sfrac1p}.$$ Since $|a-b| = |b-a|$ for all $a,b \in \R$, $d_p(\bar y, \bar x) = d_p(\bar x, \bar y)$. Therefore, $d_p(\bar y, \bar x)$ satisfies the symmetry property of metric spaces.
	
	To prove the Triangle inequality, we first prove the Cauchy-Schwarz inequality for this system. Fix $\epsilon \ge 0$ to be chosen and consider the product $\epsilon \bar x \cdot \sfrac{\bar y}{\epsilon}$. By part (i) of this problem we have $$\epsilon \bar x \cdot \frac{\bar y}{\epsilon} \leq \frac{\epsilon^p|\bar x|_p^p}{p} + \frac{|\bar y|_q^q}{\epsilon^q q}.$$ We wish to find the value of epsilon for which this upper bound is minimized. Taking the derivative with respect to epsilon and setting that value equal to zero gives
	\[\begin{split}
		0 &= \frac{p\epsilon^{p-1}|\bar x|_p^p}{p} + \frac{-q|\bar y|_q^q}{\epsilon^{q+1}q} \\
		&=\epsilon^{p-1}|\bar x|_p^p - |\bar y|_q^q\epsilon^{-q-1}.
	\end{split}
	\]
	Rearranging and solving for epsilon gives
	\[\begin{split}
		|\bar y|_q^q\epsilon^{-q-1} &= \epsilon^{p-1}|\bar x|_p^p,\\
		\frac{|\bar y|_q^q}{|\bar x|_p^p} &= \epsilon^{p-1}\epsilon^{q+1},\\
		\left ( \frac{|\bar y|_q^q}{|\bar x|_p^p} \right )^\frac{1}{p+q} &= \epsilon.
	\end{split}
	\]
	Using eq 3, this becomes $$\left ( \frac{|\bar y|_q^q}{|\bar x|_p^p} \right )^\frac{1}{pq} = \epsilon.$$ Using this value for epsilon in our original inequality gives and simplifying gives
	\[\begin{split}
		\epsilon \bar x \cdot \frac{\bar y}{\epsilon} &\leq \left [ \left ( \frac{|\bar y|_q^q}{|\bar x|_p^p} \right )^\frac{1}{pq} \right ]^p \frac{|\bar x|_p^p}{p} + \left[ \left ( \frac{|\bar y|_q^q}{|\bar x|_p^p} \right )^\frac{1}{pq} \right ] ^{-q} \frac{|\bar y|_q^q}{q}\\
		&= \left ( \frac{|\bar y|_q^q}{|\bar x|_p^p} \right )^\frac{1}{q} \frac{|\bar x|_p^p}{p} + \left ( \frac{|\bar y|_q^q}{|\bar x|_p^p} \right )^\frac{-1}{p} \frac{|\bar y|_q^q}{q}\\
		&= \left ( \frac{|\bar y|_q^{q/q}}{|\bar x|_p^{p/q}} \right ) \frac{|\bar x|_p^p}{p} + \left ( \frac{|\bar y|_q^{-q/p}}{|\bar x|_p^{-p/p}} \right ) \frac{|\bar y|_q^q}{q}\\
		&= \frac{|\bar y|_q|\bar x|_p^{p-p/q}}{p} + \frac{|\bar x|_p|\bar y|_q^{q-q/p}}{q}\\
	\end{split}
	\]
	To simplify further consider a rearrangement of Young's Inequality:
	\[\begin{split}
		1 &= \frac{1}{p} + \frac{1}{q}\\
		1-\frac{1}{p} &= \frac{1}{q}\\
		q-\frac{q}{p} &= 1.
	\end{split}
	\]
	By similar steps, the same inequality holds if the positions of $p$ and $q$ are swapped. Continuing with the simplification gives
	\[\begin{split}
		\epsilon \bar x \cdot \frac{\bar y}{\epsilon} &\leq \frac{|\bar y|_q|\bar x|_p^{1}}{p} + \frac{|\bar x|_p|\bar y|_q^{1}}{q}\\
		&= |\bar y|_q|\bar x|_p \left ( \frac{1}{p} + \frac{1}{q} \right )\\
		&= |\bar y|_q|\bar x|_p.
	\end{split}
	\]
	To prove the triangle inequality for $d_p$, we start by proving $|\bar x + \bar y|_p \leq |\bar x_p| + |\bar y_p|$ for any $\bar x, \bar y \in \R^N$. Let $|x_i + y_i|^p = (x_i+y_i)z_i$. For this to hold $z_i = |x_i+y_i|^{p-1}sign(x_i+y_i)$ where \[sign(x) = 
	\begin{cases}
		0 \quad &\text{ if } x =0\\
		1 \quad &\text{ if } x>0\\
		-1 \quad &\text { if } x<0.
	\end{cases}\]
	Thus, 
	\[\begin{split}
		|\bar x + \bar y|_p^p &= |x_1 + y_1|^p + \dotsc + |x_1 + y_1|^p\\
		&=(x_1 + y_1)z_1 + \dotsc + (x_1 + y_1)z_N\\
		&= (x_1z_1 + \dotsc + x_Nz_N) + (y_1z_1+ \dotsc + y_Nz_N)\\
		&= \bar x \cdot \bar z + \bar y \cdot \bar z\\
		&\leq |\bar x|_p|\bar z|_q + |\bar y|_p|\bar z|_q\\
		&= (|\bar x|_p| + |\bar y|_p)|\bar z|_q.
	\end{split}\]
	We now compute $|\bar z|_q$.
	\[
	\begin{split}
		|\bar z|_q &= \left ( \sum_{i=1}^{N} \left [|x_i+y_i|^{p-1}sign(x_i+y_i) \right ] ^q \right ) ^{1/q}\\
		&= \left ( \sum_{i=1}^{N} |x_i+y_i|^{q(p-1)} \right ) ^{1/q}.
	\end{split}\]
	The $sign(x_i+y_i)$ terms can be removed within the absolute value as $|-1| = |1|$ and if $sign(x_i+y_i) = 0$, $|x_i+y_i|^{p-1} = 0$ as well. Thus, the values are unchanged. By eq 3, 
	\[\begin{split}
		qp-q &= (p+q) - q\\
		&= p.
	\end{split}\]
	Thus,
	\[
	\begin{split}
		|\bar z|_q &= \left ( \sum_{i=1}^{N} |x_i+y_i|^{p} \right ) ^{1/q}\\
		&= |\bar x + \bar y|_p^{\sfrac{p}{q}}
	\end{split}\]
	Returning to our expression for $|\bar x + \bar y|_p^p$, 
	\[\begin{split}
		|\bar x + \bar y|_p^p &\leq (|\bar x|_p| + |\bar y|_p)|\bar z|_q\\
		&=(|\bar x|_p| + |\bar y|_p) |\bar x + \bar y|_p^{\sfrac{p}{q}}.
	\end{split}\]
	Thus,
	\[\begin{split}
		|\bar x|_p| + |\bar y|_p &\geq \frac{|\bar x + \bar y|_p^p}{|\bar x + \bar y|_p^{\sfrac{p}{q}}}\\
		&= |\bar x + \bar y|_p^{p-\sfrac{p}{q}}\\
		&= |\bar x + \bar y|_p.
	\end{split}\]
	Since $p-\sfrac{p}{q} = 1$ as shown in eq 3, $$|\bar x|_p| + |\bar y|_p \geq|\bar x + \bar y|_p.$$
	Consider again $d_p$. 
	\[\begin{split}
		d_p(\bar x, \bar z) &= |\bar x - \bar z|_p\\
		&= |(\bar x -\bar y) + (\bar y - \bar z)|_p.
	\end{split}\]
	By the above inequality,
	\[\begin{split}
		d_p(\bar x, \bar z) &\leq |\bar x -\bar y|_p + |\bar y - \bar z|_p\\
		&\leq d_p(\bar x, \bar y) + d_p(\bar y, \bar z).
	\end{split}\]
	Thus, $d_p$ satisfies the triangle inequality, and is therefore a metric.
\end{solution}	
\newpage




\begin{prob}{6}  % `prob' starts the (custom made, above) problem
			%  and the 
Suppose that $(X,d)$ is a metric space and $x,y,z\in X$.  Show that
\[
	d(x,z) \geq |d(x,y) - d(y,z)|.
\]
\end{prob}
%Uncomment the lines below to solve the problem
%\begin{solution}
%This is a very elegant solution.
%\end{solution}
%\newpage





\begin{prob}{7}  % `prob' starts the (custom made, above) problem
			%  and the 
Show that the following functions are continuous:
\begin{enumerate}[(i)]

	\item $\psi: \R \to \R$ defined by $\psi(x) = 2x^3 + 1$.
	
	\item $E: C(\R) \to \R$ defined by $E(f) = f(0)$

	\item $\cS: C(\R) \to C(\R)$ defined by $\cS(f) = f^2$
	
\end{enumerate}
You may find it helpful to note that $a^3 - b^3 = (a-b) (a^2 + ab + b^2)$.  Also, for (iii), please show that $\cS$ is well-defined; that is, show that $f^2 \in C(\R)$.
\end{prob}
%Uncomment the lines below to solve the problem
%\begin{solution}
%This is a very elegant solution.
%\end{solution}
%\newpage





\begin{prob}{8}  % `prob' starts the (custom made, above) problem
			%  and the 
	Suppose that $d_1, d_2$ are equivalent metrics on a set $X$.  Suppose that $(Y,d_Y)$ is a metric space and $f: (X,d_1) \to (Y, d_Y)$ is continuous.  Show that $\tilde f: (X,d_2) \to (Y,d_Y)$, defined by $\tilde f(x) = f(x)$ for all $x\in X$, is continuous.
\end{prob}
%Uncomment the lines below to solve the problem
%\begin{solution}
%This is a very elegant solution.
%\end{solution}
%\newpage








\end{document}