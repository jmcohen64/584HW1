\documentclass[11pt,letterpaper]{article}
\usepackage[top=.5in,textheight=9in]{geometry}
\usepackage{amsmath, amsthm, amssymb}
\usepackage{enumerate}
\usepackage{xfrac}


% Everything after a % sign is commented out.
% This is sometimes useful to write notes to yourself
% or to add spacing in the tex file so that it is easier
% to read.


% Some useful `macros'
% % % Feel free to define your own!
\newcommand{\C}{\mathbb{C}}
\newcommand{\N}{\mathbb{N}}
\newcommand{\R}{\mathbb{R}}
\newcommand{\Z}{\mathbb{Z}}

\newcommand{\cS}{\mathcal S}


% Here is a pretty way to write down the problem
\newtheorem{innerprob}{Problem}
\newenvironment{prob}[1]
  {\renewcommand\theinnerprob{#1}\innerprob}
  {\endinnerprob}
% Here is a pretty way to wrote down the solution
\newenvironment{solution}
  {\renewcommand\qedsymbol{}\begin{proof}[Solution]}
  {\end{proof}\bigskip}

\setlength\parindent{0cm}
\setlength\parskip{5pt plus 1pt minus 1pt}


\title{Assignment \#1\\Math 584A}
\author{
	John Cohen
	}
\date{Due September 10th at 1 am (via Gradescope)}









\begin{document}

\maketitle

For uploading to Gradescope, it will be easiest to put each solution on a different page.  The code for this is commented out in the tex file.

% Don't write anything between \begin{document}
% and \maketitle or it will show up before your name
% and the rest of the title stuff.



%State the problem
%\begin{prob}{PROBLEM #}
%  WRITE PROBLEM
%\end{prob}

% Note prob is custom-made above will



\begin{prob}{1}  % `prob' starts the (custom made, above) problem
			%  and the 
Fix any two points $x,y\in \R^n$.  Show that
\[
	\lim_{p\to\infty} d_p(x,y) = d_\infty(x,y).
\]
\end{prob}
%Uncomment the lines below to solve the problem
\begin{solution}
	Let $i\leq n$ be defined such that $|x_i-y_i| \geq |x_m-y_m|$ for all $m\leq n$. By definition, $|x_i-y_i| = d_\infty(x,y)$. Notice that $$|x_i-y_i| \leq |x_1-y_1| + \dotsc + |x_i-y_i|+ \dotsc + |x_n-y_n|\leq n|x_i-y_i|.$$ Thus, $$|x_i-y_i|^p \leq |x_1-y_1|^p + \dotsc + |x_i-y_i|^p+ \dotsc + |x_n-y_n|^p\leq n|x_i-y_i|^p.$$ By taking each expression in the inequality to the $1/p$ power, we get $$(|x_i-y_i|^p)^{\frac{1}{p}} \leq (|x_1-y_1|^p + \dotsc + |x_i-y_i|^p+ \dotsc +|x_n-y_n|^p)^{\frac{1}{p}} \leq (n|x_i-y_i|^p)^{\frac{1}{p}},$$ and
	\begin{equation}\label{e1}
		|x_i-y_i| \leq d_p(x,y) \leq n^\frac{1}{p}|x_i-y_i|.
	\end{equation}  
	Taking the limit as p approaches infinity gives $$\lim_{p\to\infty}|x_i-y_i| \leq \lim_{p\to\infty}d_p(x,y) \leq \lim_{p\to\infty}n^\frac{1}{p}|x_i-y_i|.$$ Thus, $$d_\infty(x,y) \leq \lim_{p\to\infty}d_p(x,y) \leq d_\infty(x,y).$$ Therefore, by the Squeeze Theorem,
	\[
	\lim_{p\to\infty} d_p(x,y) = d_\infty(x,y).
	\]
\end{solution}
\newpage






\begin{prob}{2}  % `prob' starts the (custom made, above) problem
			%  and the 
Fix $p_1,p_2 \in [1,\infty]$ such that $p_1 < p_2$ (we allow the case $p_2 = \infty$).  Find constants $\underline C, \overline C >0$ such that
\[
	\underline C d_{p_1}(\bar x, \bar y)
		\leq d_{p_2}(\bar x, \bar y)
		\leq \overline C d_{p_1}(\bar x, \bar y),
\]
for all $x,y \in \R^N$.  Hint: you may find it helpful to do the case $p_1 = 1$ and $p_2=\infty$ first.
\end{prob}
%Uncomment the lines below to solve the problem
\begin{solution}
	The corresponding $d_{p_1}(\bar x, \bar y)$ for $p_1$. Let $d_\infty(\bar x, \bar y) = |x_i-y_i|$ for some $i\leq N$. By \eqref{e1}, we see that $$n^{-\frac{1}{p_1}}d_{p_1}(x,y) \leq |x_i-y_i|.$$ By setting $\underline C = n^{-\frac{1}{p_1}}$, we get $\underline C d_{p_1}(\bar x,\bar y) \leq d_\infty(\bar x, \bar y)$. Since the choice of $p_1\geq 1$ was arbitrary, this is valid for any $d_p(\bar x, \bar y)$.
	
	For the same $d_\infty(\bar x, \bar y)$, factoring out $|x_i-y_i|^{p_1}$ from the expression for $d_{p_1}(\bar x, \bar y)$ gives 
	\[\begin{split}
	d_{p_1}(\bar x, \bar y) &= \left [ |x_i-y_i|^{p_1} \left ( \frac{|x_1-y_1|^{p_1}}{|x_i-y_i|^{p_1}} + \dotsc + \frac{|x_i-y_i|^{p_1}}{|x_i-y_i|^{p_1}} + \dotsc \frac{|x_N-y_N|^{p_1}}{|x_i-y_i|^{p_1}} \right ) \right ] ^{1/{p_1}} \\
	&=|x_i-y_i| \left [ \left ( \frac{|x_1-y_1|}{|x_i-y_i|} \right )^{p_1} + \dotsc + 1 + \dotsc \left ( \frac{|x_N-y_N|}{|x_i-y_i|} \right )^{p_1} \right ] ^{1/{p_1}}.
	\end{split}
	\]
	Let $a_m = \frac{|x_m-y_m|}{|x_i-y_i|}$ for each $m\leq N$. Since $|x_i-y_i| \geq |x_m-y_m|$, $a_m<1$ for all $m \leq N$. Thus, 
	\begin{equation}\label{e2}
		d_{p_1}(\bar x, \bar y) = |x_i-y_i| (a_1^{p_1} + \dotsc + 1 + \dotsc + a_N^{p_1})^{1/p_1}.
	\end{equation}
	Since we set $d_\infty(\bar x, \bar y) = |x_i-y_i|$, $$d_{p_1}(\bar x, \bar y) = d_\infty(\bar x, \bar y) (a_1^{p_1} + \dotsc + 1 + \dotsc + a_N^{p_1})^{1/p_1}.$$ Because $a_1^{p_1} + \dotsc + 1 + \dotsc + a_N^{p_1} > 1$, $(a_1^{p_1} + \dotsc + 1 + \dotsc + a_N^{p_1})^{1/p_1} > 1$. Therefore, $$d_\infty(\bar x, \bar y)\leq d_{p_1}(\bar x, \bar y).$$ Thus, 
	\begin{equation}\label{e3} 
		\underline C d_{p_1}(\bar x, \bar y) \leq d_\infty(\bar x, \bar y)\leq \overline C d_{p_1}(\bar x, \bar y) 
	\end{equation}	
	where $\underline C = n^{-\frac{1}{p_1}}$ and $\overline C = 1$.
	
	Consider the $p_2$ metric on the same vectors $\bar x$ and $\bar y$. Equation \eqref{e2} for $p_2$ gives $$d_{p_2}(\bar x, \bar y) = |x_i-y_i| (a_1^{p_2} + \dotsc + 1 + \dotsc + a_N^{p_2})^{1/p_2}.$$ Since $p_2 > p_1$ and $a_m < 1$, $a_m^{p_2}\leq a_m^{p_1}$ and $$(a_1^{p_2} + \dotsc + 1 + \dotsc + a_N^{p_2})^{1/p_2} \leq a_1^{p_1} + \dotsc + 1 + \dotsc + a_N^{p_1})^{1/p_1}.$$ Therefore, $$d_{p_2}(\bar x, \bar y) \leq d_{p_1}(\bar x, \bar y).$$ Combining this with \eqref{e3} gives $$n^{-\frac{1}{p_1}}d_{p_1}(\bar x, \bar y) \leq d_\infty(\bar x, \bar y) \leq d_{p_2}(\bar x, \bar y) \leq d_{p_1}(\bar x, \bar y).$$ Thus, $$\underline C d_{p_1}(\bar x, \bar y) \leq d_{p_2}(\bar x, \bar y)
	\leq \overline C d_{p_1}(\bar x, \bar y)$$ where $\underline C = n^{-\frac{1}{p_1}}$ and $\overline C = 1$.
	
\end{solution}
\newpage






\begin{prob}{3}  % `prob' starts the (custom made, above) problem
			%  and the 
Consider the metric space $(X, d_{\rm disc})$, where $X$ is a nonempty set.  Suppose that $(x_n)_n$ is a sequence in $X$.  Show that
\[
	\lim_{n\to\infty} x_n = x
\]
if and only if there exists $N$ such that $x_n = x$ for all $n \geq N$.
\end{prob}
%Uncomment the lines below to solve the problem
\begin{solution}
	Fix any $\epsilon$ such that $0 < \epsilon < 1$. Since $\lim_{n\to\infty} x_n = x$, there must exist $N \in \N $ such that $d_{disc}(x_n,x)< \epsilon$ whenever $n\geq N$. Since we are in the discrete metric, $d_{disc}(x_n,x)< 1$ only when $x_n = x$. Therefore, $x_n = x$ for all $n\geq N$.
	
	Assume there exists some $N\in \N$ such that for all $n\geq N$, $x_n = n$. Thus, for all $n\geq N$, $d_{disc}(x_, x)  = 0$. Therefore, $d_{disc}(x_, x) < \epsilon$ for any $\epsilon > 0$. Thus, $\lim_{n\to\infty} x_n = x$.
\end{solution}
\newpage



\begin{prob}{4}  % `prob' starts the (custom made, above) problem
			%  and the 
Prove Young's inequality: for any $p,q >1$ such that
\[
	1
		= \frac{1}{p} + \frac{1}{q}
\]
for any $x,y\in \R$,
\[
	xy \leq \frac{|x|^p}{p} + \frac{|y|^q}{q}.
\]
Hint: for any fixed $y \geq 0$, consider the function $f: [0,\infty) \to \R$, defined by
\[
	f(x) = \frac{x^p}{p} + \frac{y^q}{q} - xy,
\]
and show that the minimum of $f$ is zero.
\end{prob}
%Uncomment the lines below to solve the problem
%\begin{solution}
%This is a very elegant solution.
%\end{solution}
%\newpage



\begin{prob}{5}  % `prob' starts the (custom made, above) problem
			%  and the 
Fix $N\geq 1$ and $p>1$.  Let $q$ be as in problem 4. Show the following:
\begin{enumerate}[(i)]

	\item Show that, for any $\bar x, \bar y \in \R^N$, we have
		\[
			x\cdot y \leq \frac{|x|_p^p}{p} + \frac{|y|_q^q}{q}.
		\]
		Recall that
		\[
			|x|_p = \left(|x_1|^p + \cdots + |x_N|^p\right)^{\sfrac1p},
		\]
		and similarly for $|\cdot|_q$.  
	
	\item Show that $d_p$ is a metric on $\R^N$.  You may find it helpful to write, for any $i=1,\dots, N$,
		\[
			|x_i - y_i|^p = (x_i - y_i) z_i
		\]
		for a well-chosen $z_i$.
	
\end{enumerate}
\end{prob}
%Uncomment the lines below to solve the problem
%\begin{solution}
%This is a very elegant solution.
%\end{solution}
%\newpage




\begin{prob}{6}  % `prob' starts the (custom made, above) problem
			%  and the 
Suppose that $(X,d)$ is a metric space and $x,y,z\in X$.  Show that
\[
	d(x,z) \geq |d(x,y) - d(y,z)|.
\]
\end{prob}
%Uncomment the lines below to solve the problem
%\begin{solution}
%This is a very elegant solution.
%\end{solution}
%\newpage





\begin{prob}{7}  % `prob' starts the (custom made, above) problem
			%  and the 
Show that the following functions are continuous:
\begin{enumerate}[(i)]

	\item $\psi: \R \to \R$ defined by $\psi(x) = 2x^3 + 1$.
	
	\item $E: C(\R) \to \R$ defined by $E(f) = f(0)$

	\item $\cS: C(\R) \to C(\R)$ defined by $\cS(f) = f^2$
	
\end{enumerate}
You may find it helpful to note that $a^3 - b^3 = (a-b) (a^2 + ab + b^2)$.  Also, for (iii), please show that $\cS$ is well-defined; that is, show that $f^2 \in C(\R)$.
\end{prob}
%Uncomment the lines below to solve the problem
%\begin{solution}
%This is a very elegant solution.
%\end{solution}
%\newpage





\begin{prob}{8}  % `prob' starts the (custom made, above) problem
			%  and the 
	Suppose that $d_1, d_2$ are equivalent metrics on a set $X$.  Suppose that $(Y,d_Y)$ is a metric space and $f: (X,d_1) \to (Y, d_Y)$ is continuous.  Show that $\tilde f: (X,d_2) \to (Y,d_Y)$, defined by $\tilde f(x) = f(x)$ for all $x\in X$, is continuous.
\end{prob}
%Uncomment the lines below to solve the problem
%\begin{solution}
%This is a very elegant solution.
%\end{solution}
%\newpage








\end{document}